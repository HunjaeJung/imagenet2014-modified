\documentclass[10pt,twocolumn,letterpaper]{article}

\usepackage{cvpr}
\usepackage{times}
\usepackage{epsfig}
\usepackage{graphicx}
\usepackage{amsmath}
\usepackage{amssymb}

% Include other packages here, before hyperref.

% If you comment hyperref and then uncomment it, you should delete
% egpaper.aux before re-running latex.  (Or just hit 'q' on the first latex
% run, let it finish, and you should be clear).
\usepackage[breaklinks=true,bookmarks=false]{hyperref}

\cvprfinalcopy % *** Uncomment this line for the final submission

\def\cvprPaperID{****} % *** Enter the CVPR Paper ID here
\def\httilde{\mbox{\tt\raisebox{-.5ex}{\symbol{126}}}}

% Pages are numbered in submission mode, and unnumbered in camera-ready
%\ifcvprfinal\pagestyle{empty}\fi
\setcounter{page}{4321}
\begin{document}

%%%%%%%%% TITLE
\title{A Dynamic Approach to ImageNet Competition Challenge}

\author{beomjun shin\\
Industrial Engineering\\
{\tt\small lucidus21@gmail.com}
% For a paper whose authors are all at the same institution,
% omit the following lines up until the closing ``}''.
% Additional authors and addresses can be added with ``\and'',
% just like the second author.
% To save space, use either the email address or home page, not both
\and
hunjae jung\\
Computer Science, mathematics\\
{\tt\small hunjaeme@gmail.com}
\and
youngjin kim\\
Mechanical Engineering\\
{\tt\small youngj0908@gmail.com}
}

\maketitle
%\thispagestyle{empty}

%%%%%%%%% BODY TEXT
\section{Introduction}

In this project, we would like to challenge previous ImageNet 2014 competition by using deep learning framework called Caffe.
ImageNet 2014 competition is one of the largest and the most challenging computer vision challenge.
In this competition, there were 3 problems which are (1) image classification, (2) object detection, and (3) several objects detection.
To solve these problems, we would mainly focus on the techniques called Convolutional Neural Network(CNN) for classification, Regions with Convolutional Neural Network(RCNN) for object detection, and several computer vision skills what we?ve learned in class to optimize our algorithm.
Caffe, which is deep learning framework, let us use CNN/RCNN easily and provides several API function for deep learning algorithms.
When we success on image classification and object detection tasks, we would make simple application to show our works.
When the user upload there own picture on the application, the application will provides related images depending on the result of image classification tasks.
And based on this project experience, we hope we could challenge on ImageNet 2016 competition or create a novel application by using this state-of-the-art performance.
\LaTeX\ Tiny ImageNet Challenge

%-------------------------------------------------------------------------
\section{Team Members}

List up your team members, and (planned) role of each member

\section{Problem}

\section{References}

\section{Methods}

\section{Evaluation}


List and number all bibliographical references in 9-point Times,
single-spaced, at the end of your paper. When referenced in the text,
enclose the citation number in square brackets, for
example~\cite{Authors14}.  Where appropriate, include the name(s) of
editors of referenced books.

{\small
\bibliographystyle{ieee}
\bibliography{egbib}
}

\end{document}
